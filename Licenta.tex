\documentclass[a4paper]{article}

%% Language and font encodings
\usepackage[english]{babel}
\usepackage[utf8x]{inputenc}
\usepackage[T1]{fontenc}

%% Sets page size and margins
\usepackage[a4paper,top=3cm,bottom=2cm,left=3cm,right=3cm,marginparwidth=1.75cm]{geometry}



\title{UNIVERSITATEA ALEXANDRU IOAN CUZA IAŞI
          Battle of Gods
}

\author{Adrian Dones}

\begin{document}
\maketitle



\section{Cuprins}

1.Introducere\\
2.Descriere\\
3.Dezvoltare

\section{Introducere}
 Jocurile au fost mereu o sursă de relaxare pentru oameni. "Battle of Gods" reprezintă un joc de cărți. Acest joc stimulează imaginația strategică a concurenților pentru a găsi posibilități ca să-și învingă adversarii. 

\section{Descriere}
La intrarea în aplicație, utilizatorul va putea alege dintre dificultățile pe care adversarul său le poate avea pe durata jocului, acestea fiind easy, medium și hard. Odată selectată dificultatea, meciul va începe. Fiecare jucător va avea un pachet de 20 cărți care va conține cărți de tip "Spell", "Trap" și "Monster". Tabla de joc permite fiecărui jucător să poată pune atât 3 cărți de tip "Monster", cât și 3 cărți de tip"Spell"/"Trap".  
 Pe parcursul jocului, adversarii pot pune pe tabla de joc tipurile de cărți prezentate mai sus pentru a putea ataca adversarul și pentru a-i reduce la zero punctele de viață. La final, câștigătorul va primi o carte random nouă în pachetul său. 
\section{Dezvoltare}
 Aplicația este utilizabilă pe platforma Android. Doresc să o dezvolt prin a introduce mai multe opțiuni de joc, de exemplu "match-duel", prin care cei doi jucători pot juca 3 meciuri și cel care are două meciuri din 3 câștigate, este învingătorul. De asemenea, doresc să o dezvolt prin a introduce mai multe tipuri de cărți precum "fusion-monster", prin care pot combina două cărți de tip "Monster" pentru a obține o nouă carte cu funcții noi.
\end{document}